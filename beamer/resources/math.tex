\DeclareMathOperator{\AND}{AND}
\DeclareMathOperator{\atantwo}{atan2}
\DeclareMathOperator{\card}{card}
\DeclareMathOperator{\sgn}{sgn}
\DeclareMathOperator{\sinc}{sinc}
\DeclareMathOperator{\Var}{Var}
\DeclareMathOperator{\XOR}{XOR}

\newcommand{\trans}{\ensuremath{^\top}}%

\newcommand{\vecG}{\ensuremath{\mathbf{g}}}%
\newcommand{\vecW}{\ensuremath{\mathbf{w}}}%
\newcommand{\vecX}{\ensuremath{\mathbf{x}}}%
\newcommand{\vecY}{\ensuremath{\mathbf{y}}}%
\newcommand{\vecZ}{\ensuremath{\mathbf{z}}}%
\newcommand{\netout}{\ensuremath{\hat{y}}}%
\newcommand{\nettarget}{\ensuremath{y}}%
\newcommand{\exampleloss}{\ensuremath{L}}%
\newcommand{\lossfunc}{\ensuremath{J}}%
\newcommand{\lossfuncCompl}{\ensuremath{\lossfunc(\vecW, \vecX, y)}}
\newcommand{\learnrate}{\ensuremath{\eta}}%

\newcommand{\ce}{\ensuremath{\mathrm{e}}}%
\newcommand{\ci}{\ensuremath{\mathrm{i}}}%
\newcommand{\iu}{\mathrm{i}\mkern1mu}% ??

\newcommand{\tanhFrac}[1]{\ensuremath{\dfrac{2}{1 \! + \! \ce^{-2\left( #1 \right)}} - 1 }}
\newcommand{\tanhFracN}[2]{\ensuremath{\dfrac{2 \cdot #1}{1 \! + \! \ce^{-2\left( #2 \right)}} - 1 }}


\pgfmathdeclarefunction{gauss}{2}{%
    \pgfmathparse{1/(#2*sqrt(2*pi))*exp(-((x-#1)^2)/(2*#2^2))}%
}

% https://mylatexnotes.wordpress.com/2017/05/08/plots-how-to-select-first-n-rows-of-data-to-plot/
% Style to select only points from #1 to #2 (inclusive)
\pgfplotsset{select coords between index/.style 2 args={%
            x filter/.code={%
                    \ifnum\coordindex<#1\def\pgfmathresult{}\fi%
                    \ifnum\coordindex>#2\def\pgfmathresult{}\fi%
                }
        }}